\usepackage[round,authoryear]{natbib}
\usepackage{paralist}
\usepackage{pgfpages}
\usepackage{mathtools}
\usepackage{amsmath}
\usepackage{amssymb}
\usepackage{xspace}        
\usepackage{graphicx}

\bibliographystyle{jss}

\usefonttheme[onlymath]{serif}
% use article-like math letters, from http://tex.stackexchange.com/questions/34265/how-to-get-beamer-math-to-look-like-article-math

\newcommand\prob[1]{\mathbb{P}\left[{#1}\right]}
\newcommand\expect[1]{\mathbb{E}\left[{#1}\right]}
\newcommand\var[1]{\mathrm{Var}\left[{#1}\right]}
\newcommand\dist[2]{\mathrm{#1}\left(#2\right)}
\newcommand\dlta[1]{{\Delta}{#1}}
\newcommand\lik{\mathcal{L}}
\newcommand\loglik{\ell}
\newcommand\R{\mathbb{R}}
\newcommand\Rzero{\mathfrak{R}_0}
\newcommand\argmax{\mathop{\mathrm{argmax}}}
\newcommand\argmin{\mathop{\mathrm{argmin}}}

\newcommand\code[1]{\texttt{#1}}
\newcommand\package[1]{\textbf{#1}}

\newcommand\link[2]{\href{#1}{#2}}
\newcommand{\doi}[1]{\link{https://doi.org/#1}{\texttt{doi:~{#1}}}}

\newcommand{\scinot}[2]{#1{\times}10^{#2}}
\newcommand{\dd}[1]{\mathrm{d}{#1}}
\newcommand{\pd}[3][]{%
  \def\ord{#1} \ifx\ord\empty%
  \frac{\partial{#2}}{\partial{#3}}%
  \else \frac{\partial^{#1}{#2}}{\partial{#3}^{#1}}%
  \fi
}
\newcommand{\deriv}[3][]{%
  \def\ord{#1} \ifx\ord\empty%
  \frac{\dd{#2}}{\dd{#3}}%
  \else \frac{\dd^{#1}{#2}}{\dd{#3}^{#1}}%
  \fi
}

\newcommand\Rlanguage{\textsf{R}\xspace}

\newcommand\question{{\bf Question}}
\newcommand\mysolution{{\bf Solution}}
\newcounter{Qcounter}
\newcommand\myquestion{{\stepcounter{Qcounter} Question \CHAPTER.\theQcounter}}
\newcounter{Ecounter}
\newcommand\myexercise{{\stepcounter{Ecounter} Exercise \CHAPTER.\theEcounter}}

\newcommand\myexample{{\bf Example}}
\newcommand\mydot{{\,\cdot\,}}
\newcommand\myref[1]{\m{#1}}

\newcommand\Rspace{\mathcal{R}}

\renewcommand\vec[1]{\boldsymbol{\mathrm{#1}}}
\newcommand\vect[1]{\vec{#1}}
\newcommand\mat[1]{\mathbb{#1}}
\newcommand\pr{\mathbb{P}}
\newcommand\E{\mathbb{E}}

\newcommand\profileloglik[1]{\ell^\mathrm{profile}_#1}
\newcommand\Real{\mathbb{R}}

\newcommand\bi{\begin{itemize}}
\newcommand\ei{\end{itemize}}

\newcommand\normal{\mathrm{normal}}

\newcommand\iid{\mathrm{iid}}
\newcommand\MVN{\mathrm{MVN}}
\newcommand\SE{\mathrm{SE}}

\newcommand\pval{\mathrm{pval}}
% \newcommand\var{\mathrm{Var}}
\newcommand\sd{\mathrm{SD}}
\newcommand\sdSample{\mathrm{sd}}
\newcommand\varSample{\mathrm{var}}
\newcommand\cov{\mathrm{Cov}}
\newcommand\covSample{\mathrm{cov}}
\newcommand\corSample{\mathrm{cor}}
\newcommand\cor{\mathrm{Cor}}
\newcommand\given{{\, | \,}}
\newcommand\param{\,;}
\newcommand\params{\param}
\newcommand\equals{{ \, = \, }}
\newcommand\transpose{{\raisebox{0.5mm}{\mbox{\scriptsize \textsc{t}}}}}
%\newcommand\transpose{\scriptsize{T}}
\newcommand\mycolon{{\hspace{0.5mm}:\hspace{0.5mm}}}
%\newcommand\mycolon{\,{:}\,}
\newcommand\myemph[1]{{\textbf{#1}}}
\newcommand\mymathenv[1]{\textcolor{blue}{#1}}


%% following a style where all math is in blue
%% not currently used by SISMID slides
\newcommand\mymath[1]{\begin{math}\textcolor{blue}{#1}\end{math}}
\newcommand\m[1]{\mymath{#1}}
\newcommand\mydisplaymath[1]{\begin{displaymath}\textcolor{blue}{#1}\end{displaymath}}
\newcommand\myeqnarray[1]{\textcolor{blue}{\begin{eqnarray*}#1 \end{eqnarray*}}}
\newcommand\myspace{\quad}
\newcommand\altdisplaymath[1]{\vspace{1mm}\textcolor{blue}{\begin{math}\displaystyle #1 \end{math}}\vspace{1mm}}


%% not currently used in SISMID slides.
%% a plain and space-efficient list useful for dense slides
%% the compactenum and compactitem environments from paralist are probably prefer%% able
%% \newcommand\enumerateSpace{\hspace{2mm}}
%% \newenvironment {myitemize} {
%%                  \begin{list}{\textcolor{black}{$\bullet$} \hfill}
%%                  {\setlength{\labelwidth}{0.3 cm}
%%                   \setlength{\leftmargin}{0.15cm}
%%                   \setlength{\itemindent}{0.15cm}
%%                   \setlength{\labelsep}{0cm}
%%                   \setlength{\parsep}{0.2 ex}
%%                   \setlength{\itemsep}{1 mm}
%%       \setlength{\topsep}{0.0cm}}} %space between title and 1st item
%%    {\end{list}}

\setlength{\parskip}{0mm}
\setlength{\parindent}{0mm}

\newcommand\negBeforeCode{}
\newcommand\negAfterCode{}

\mode<beamer>{\usetheme{AnnArbor}}
\mode<beamer>{\setbeamertemplate{footline}}
\mode<beamer>{\setbeamertemplate{footline}[frame number]}
\mode<beamer>{\setbeamertemplate{frametitle continuation}[from second][\insertcontinuationcountroman]}
\mode<beamer>{\setbeamertemplate{navigation symbols}{}}

\mode<handout>{\pgfpagesuselayout{2 on 1}[letterpaper,border shrink=5mm]}
